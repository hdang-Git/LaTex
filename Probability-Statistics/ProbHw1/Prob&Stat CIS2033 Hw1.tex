\documentclass[fleqn]{article}

\usepackage[]{amsmath, amssymb} 	%mathformatting and symbols
\setlength{\mathindent}{0pt}
%\usepackage{graphicx}           % insert graphics
%\usepackage{eulervm, bookman}   % fonts for math & symbols
\usepackage{fullpage}          % fullpage margins
\usepackage{verbatim}
\begin{document}
\title{Probability \& Statistics CIS 2033: Hw 1}
\date{January 24, 2016}
\author{Hai Dang}

\maketitle

\begin{enumerate}
	\item[2.4.]
 	Among employees of a certain firm, 70\% know C/C++, 60\% know Fortran, and 50\% know
	both languages. What portion of programmers
	\begin{comment}	
		\\
		\begin{flalign*}
 		Let\;F:=\;know\;Fortran,\;C:=\;know\;C/C+\!+,\;and\;B :=\;know\;both
 		\end{flalign*}
 	\end{comment}
 	\begin{enumerate}
 		\item %a
 			does not know Fortran?
 			\begin{flalign*}
 			P(not\;Fortran) = P(\overline{F}) = 1 - P(F) = 1 - 0.6 = 0.4
 			\end{flalign*}
 			%P(\bar{F}) for negation but now using \overline{}
 		\item %b
 			does not know Fortran and does not know C/C++?
 			\begin{flalign*}
 			&P(not\;Fortran\;\&\;not\;C/C+\!+) \\
 			&= P(\Omega)-[P((\overline{F} \cup (\overline{C/C+\!+}))/(both))] \\
 			&= P(\Omega)-[P(Fortran)+ (P(C/C+\!+)-P(both)] \\
 			&= 1-(0.7+0.6-0.5) = 1 - 0.8 = 0.2
 			\end{flalign*}	
 			
		\paragraph{}
		\underline R: Or is the question asking P(does not know both)? 	
 			
 			\begin{comment}
 			\begin{flalign*}
 			P(not\;Fortran\;\&\;not\;C/C+\!+) = P(\overline{both}) = 1 - 0.5 = 0.5
 			\end{flalign*}	
 			\end{comment}	
 			
 		\item %c
 			knows C/C++ but not Fortran?
 			\begin{flalign*}
 			P(C/C+\!+ \;\;/ Fortran) = P(C/C++)-P(both) = 0.7-0.5=0.2
 			\end{flalign*}
 		\item %d
 			knows Fortran but not C/C++?
 			\begin{flalign*}
 			P(Fortran\;/\;C/C+) = P(Fortran)-P(both) = 0.6 -0.5 = 0.1
 			\end{flalign*}
 		\item %e
 			If someone knows Fortran, what is the probability that he/she knows C/C++ too?
 			\begin{flalign*}
 			P(C/C+\!+\;|Fortran) = \frac{P(C/C+\!+\;\cap\;Fortran)}{P(Fortran)} 
 			= \frac{0.5}{0.6} = \frac{5}{6}
 			\end{flalign*}
 		\item %f
 			If someone knows C/C++, what is the probability that he/she knows Fortran too?
 			\begin{flalign*}
 			P(Fortran|\;C/C+\!+) = \frac{P(Fortran \;\cap\; C/C+\!+)}{P(C/C+\!+)} 
 			= \frac{0.5}{0.7} = \frac{5}{7}
 			\end{flalign*}
 	\end{enumerate}


% --------------------------------------------------------------------------------------
 	
 	\item[2.7.]
    A system may become infected by some spyware through the internet or e-mail. Seventy
    percent of the time the spyware arrives via the internet, thirty percent of the time 	    via email.
    If it enters via the internet, the system detects it immediately with probability            	0.6.
	If via e-mail, it is detected with probability 0.8. What percentage of times is this 		spyware detected?
	\begin{flalign*}
	Let\;&I:=spyware\;via\;internet,\;E:= spyware\;via\; email,\\&D:=spyware detected by system
	\end{flalign*}
	\begin{flalign*}
	&Givens: P(I)=0.7,\;P(E)=0.3, P(D|I)=0.6, P(D|E) = 0.8\\\\
	&P(D|I)= \frac{P(D\cap I)}{P(I)} \rightarrow P(D\cap I)=P(I)P(D|I) =0.6*0.7=0.42\\
	&P(D|E)= \frac{P(D\cap E)}{P(E)} \rightarrow P(D\cap E)=P(E)P(D|I)= 0.8*0.3=0.24\\
	&P(D) =P(I)P(D|I)+P(D\cap E)) = 0.66
	\end{flalign*}
 	
% --------------------------------------------------------------------------------------
	\item[2.11.]
	A computer program is tested by 5 independent tests. If there is an error, these 	 	  	tests
 	will discover it with probabilities 0.1, 0.2, 0.3, 0.4, and 0.5, respectively. 	 	 	 	Suppose that the
	program contains an error. What is the probability that it will be found
	\paragraph{}
	Given: T1 = 0.1, T2 = 0.2, T3 = 0.3, T4 = 0.4, T5 = 0.5
	\begin{enumerate}
	\item %a
	by at least one test?
	\begin{flalign*}
	&P(at\;least\;one\;test)=P(T1\cup T2\cup T3\cup T4\cup T5) 
	%= 1 - P(no error detected)
	= 1 - P(\overline{T1\cup T2\cup T3\cup T4\cup T5})\\
	&= 1 - P(\overline{T1}\cap \overline{T2}\cap \overline{T3}\cap \overline{T4}\cap 	
	\overline{T5}\cap)
	= 1 - 
	P(\overline{T1})P(\overline{T2})P(\overline{T3})P(\overline{T4})P(\overline{T5})\\
	&= 1 - P(1-T1)P(1-T2)P(1-T3)P(1-T4)P(1-T5)\\
	&= 1 - (1-0.1)(1-0.2)(1-0.3)(1-0.4)(1-0.5) = 0.8488	
	\end{flalign*}
	
	\item %b
	by at least two tests?
	\begin{flalign*}
	&P(at\;least\;two\;tests)
	= 1- P(\overline{T1\cup T2\cup T3\cup T4\cup T5})-
	P(T1 \cap \overline{T2} \cap \overline{T3} \cap \overline{T4} \cap \overline{T5})\\
	&-P(\overline{T1} \cap T2 \cap \overline{T3} \cap \overline{T4} \cap \overline{T5})
	-P(\overline{T1} \cap \overline{T2} \cap T3 \cap \overline{T4} \cap \overline{T5})
	-P(\overline{T1} \cap \overline{T2} \cap \overline{T3}  \cap T4 \cap \overline{T5})\\
	&-P(\overline{T1} \cap \overline{T2} \cap \overline{T3} \cap \overline{T4}\cap T5)
	\\
	&= P(at\;least\;one\;)-P(T1)P(\overline{T2})P(\overline{T3})P(\overline{T4})    	 		     P(\overline{T5}) -... - P(\overline{T1})P(\overline{T2})P(\overline{T3})P( 
	\overline{T4})P(T5)
	\\
	%&=1-0.8488- P(T1)P(1-T2)P(1-T3)P(1-T4)P(1-T5)-...-P(1-T1)P(1-T2)
	%P(1-T3)P(1-T4)P(1-T5)\\
	&= 0.8488 - 0.0168-0.0378-0.0648-0.1008-0.1512 = 0.4774
	\end{flalign*}

	\item %c
	by all five tests?
	\begin{flalign*}
	&P(all\;five\;tests)= {(T1\cap T2\cap T3\cap T4\cap T5)} = P(T1)P(T2)P(T3)P(T4)P(T5)
	\\
	&= (0.1)(0.2)(0.3)(0.4)(0.5)= 0.0012 
	\end{flalign*}
	
	 
	\end{enumerate}
% --------------------------------------------------------------------------------------	
	\item[2.13.]
	An important module is tested by three independent teams of inspectors. Each team 
	detects a problem in a defective module with probability 0.8. What is the probability 
	that at least one team of inspectors detects a problem in a defective module?
	\paragraph{}
	Let T1 = team 1, T2 = team 2, T3 = team 3  where P(T1) = P(T2) = P(T3) = 0.8 
	\begin{flalign*}	
	&P(at\;least\;one\;team\;detects\;a\;problem\;in\;defective\;module)\\
	&= P(T1 \cup T2 \ T3) = 1 - P(\overline{T1}\cap\overline{T2}\cap\overline{T3}) \\
	&= 1 - P(\overline{T1})^{3}
	= 1 - (1 - T1)^{3} = 1 - (1-0.8)^{3} = 1 - (.2)^{3} = 0.992
	\end{flalign*} 

% --------------------------------------------------------------------------------------	
	\newpage	
	\item[2.16.]
	A computer maker receives parts from three suppliers, S1, S2, and S3. Fifty percent 
	come from S1, twenty percent from S2, and thirty percent from S3. Among all the parts 
	supplied by S1, 5\% are defective. For S2 and S3, the portion of defective parts is 
	3\% and 6\%, respectively.
	\paragraph{}
	Givens: P(S1) = 0.5, P(S2) = 0.2, P(S3) = 0.3; 
	\\P(D \vline\;S1) = 0.05,\; P(D\vline\;S2) = 0.03, \;P(D\vline\;S3) = 0.06
	
	\begin{enumerate}
	\item %a
	What portion of all the parts is defective?
	\begin{flalign*}
	P(D) = P(S1)P(D|S1)+P(S2)P(D|S2)+P(S3)P(D|S3) = 0.5(0.05)+0.2(0.03)+0.3(0.06)=0.049
	\end{flalign*}
	\item %b
	A customer complains that a certain part in her recently purchased computer is 
	defective. What is the probability that it was supplied by S1?
	\begin{flalign*}
	P(D|S1) = \frac{P(S1|D)P(S1)}{P(D)} = \frac{(0.05)(.5)}{0.049} = 0.510
	\end{flalign*}
	\end{enumerate}
	
\end{enumerate}


\end{document}
