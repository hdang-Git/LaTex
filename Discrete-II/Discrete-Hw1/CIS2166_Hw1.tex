\documentclass[fleqn]{article}
\usepackage[]{amsmath, amssymb} 	%math formatting and symbols
\setlength{\mathindent}{0pt}
\usepackage{fullpage}          		% fullpage margins
\usepackage{verbatim}					
\usepackage{arrayjobx}				%provides array as a data structure
\usepackage{ifthen}					%for if statements
\usepackage{multido}				%looping with multiple conditions
\usepackage{amsthm} 				%used for theorems & proofs

\usepackage{algorithm}				%for writing in algorithm form
\usepackage{algpseudocode}
\newcommand{\tab}[1]{\hspace{.2\textwidth}\rlap{#1}}


\begin{document}
\title{Math Concepts in Computing II CIS 2166: Hw 1}
\date{September 2, 2016}
\author{Hai Dang}
\maketitle

\begin{enumerate}
\item[3.1.14)]
List all the steps used to search for 7 in the sequence given
in Exercise 13 for both a linear search and a binary search.
\\
1, 3, 4, 5, 6, 8, 9, 11
	%Note: arrayjobx package naming convention must not include numbers in array name
	\newarray\Values
	\readarray{Values}{1&3&4&5&6&8&9&11}
	%\checkValues(0)%
    %\Values(3)
    \begin{enumerate}
    \item %a
    	Linear Search:
    	\newcommand{\num}{7}
    	\newcommand{\length}{9} %  <= DNE
    	\newcounter{i}
    	\setcounter{i}{1}
    	\newcounter{pt}
    	\setcounter{pt}{1}
    	
    	\whiledo{\value{i} < \length }{%
    		\ifnum \Values(\the\value{i}) = \num
    			%trick for if statements since only limited to <, > , =
    			%else part can be used for != 
    			\\
    		\else %array value != 7
    			\arabic{pt}. Set counter to \arabic{i}.
    			\stepcounter{pt}
    			\newline
    			\arabic{pt}. Compare counter \arabic{i} to size of list 8.
    			\stepcounter{pt}
    			\newline
    			\arabic{pt}. Then compare 7 to \Values(\arabic{i}), the value in the array.
    			\stepcounter{pt}
    			\newline
    			\arabic{pt}. Since counter \arabic{i} $\leq$ 8 and values 7 $\neq$
    			 \Values(\arabic{i}) then increment counter by 1.
    	    	\stepcounter{i}
    	    	\stepcounter{pt}
    	    	\newline
    		\fi
    	}    	
    	\arabic{pt}. Since counter \arabic{i} > size of list 8. Exit the loop.
		\stepcounter{pt}		
		\newline    	
    	\arabic{pt}. Compare the counter \arabic{i} to the size of list 8.
    	\stepcounter{pt}
    	\newline
    	\ifnum \the\value{i} < \length - 1		
    		\arabic{pt}. Since \arabic{i} $<$ 8, number is found. Return index \the\value{i} 
    	\else
    		\arabic{pt}. Since \arabic{i} $>$ 8, number not found in array: return 0
    	\fi
	\delarray\Values
    \item %b
   		Binary Search: 	
  			\begin{enumerate}
  				\item[1.]
  				Set counter i to 1 for the lowest index.
  				\item[2.]
  				Set counter j to 8 being the size of the largest array index.
  				\item[3.]
  				Compare the counters with values 1 and 8.
  				\item[4.]
  				Since for the counters 1 $<$ 8, enter the loop.
  				\item[5.]
  				Compute midpoint as the floor of (1+8)/2 which is 4
  				\item[6.]
  				Then compare the value 7 to 5, the value in the array.
  				\item[7.]
  				Since 7 $>$ 5, set the i counter to 5.
  				\item[8.]
  				Compare the counters with values 5 and 8.
  				\item[9.]
  				Since for the counters 5 $<$ 8, enter the loop.
  				\item[10.]
  				Compute midpoint as the floor of (5 + 8)/2 which is 6.
  				\item[11.]
  				Then compare the value 7 to 8, the value in the array.
  				\item[12.]
  				Since 7 $<$ 8, set the j counter to 6.
  				\item[13.]
  				Compare the counters with values 6 and 8.
  				\item[14.]
  				Since for the counters 6 $<$ 8, enter the loop.
  				\item[15.]
  				Compute midpoint as the floor of (6 + 8)/2 which is 7.
  				\item[16.]
  				Then compare the value 7 to 9, the value in the array.
  				\item[17.]
  				Since 7 $<$ 9, set the j counter to 7.
  				\item[18.]
  				Compare the counters with values 7 and 7.
  				\item[19.]
  				Since 7 is not $<$ 7 exit the loop.
  				\item[20.]
  				Compare 7 with array value 9 at location 7.
  				\item[21.]
  				Since 7 $\neq $ 9, number not found in array: return 0.
  				
  			\end{enumerate}
    \end{enumerate}
    
\item[3.1.28)]
Specify the steps of an algorithm that locates an element
in a list of increasing integers by successively splitting
the list into four sublists of equal (or as close to equal as
possible) size, and restricting the search to the appropriate
piece.

\begin{comment}
procedure binary search (x: integer, a1, a2, . . . , an: increasing integers)
i := 1{i is left endpoint of search interval}
j := n {j is right endpoint of search interval}
while i < j
m := (i + j)/2
if x > am then i := m + 1
else j := m
if x = ai then location := i
else location := 0
return location{location is the subscript i of the term ai equal to x, or 0 if x is not found}
\end{comment}

\begin{algorithm} 
	\caption {Quaternary Search} 
	\begin{algorithmic}[1]
	\Procedure{}{$x$ : integer, $a_1, a_2, ... , a_n$ : increasing integers}
	\State $i \gets 1$
	\State $j \gets n$
	\While{$i < j$}	
	    \State $lq = \lfloor (i + j)/4 \rfloor$ 	\Comment{Lower Quarter}
		\State $mid = \lfloor (i + j)/2 \rfloor$ 	\Comment{Middle}
		\State $uq = \lfloor 3(i + j)/4 \rfloor$ 	\Comment{Upper Quarter}
		\\
		\If{$x > a[mid]$}					\Comment{Upper half, greater than the midpt}
			\If{$x > a[uq]$}				
				\State $i \gets uq+1$		\Comment{range is from i = uq+1 to j = rightmost}
		     \Else 							
		     	\State $i \gets m + 1$ 		\Comment{range is from i = m+1 to j = uq}
		     	\State $j \gets uq$
		     \EndIf
		
		\Else								\Comment{Lower half, less than and equal to midpt}
			\If{$x > a[lq]$}
				\State $i \gets lq + 1$		\Comment{range is from i = lq+1 to j = m}
				\State $j \gets m$			
			\Else
				\State $j = lq$				\Comment{range is from i = leftmost to j = lq}
			\EndIf
		\EndIf

		\\
	\EndWhile
	\\
	\If {$x = a_i$}
		\State $location \gets i$	\Comment{Set location to index i}
	%\ElsIf{$x = a_j$}
	%	\State $location \gets j$ 	\Comment{Set location to index j}
	\Else	
		\State $location \gets -1$	\Comment{x is not found in list}
	\EndIf
	\\
	\Return $location$
	\EndProcedure
	\end{algorithmic}
\end{algorithm}






\begin{algorithm}
\item[3.1.44)]    
Describe an algorithm based on the binary search for determining
the correct position in which to insert a new
element in an already sorted list.
\\
		\caption {Binary Search Insert} 
	\begin{algorithmic}[1]
	\Procedure{}{$x$ : integer, $a_1, a_2, ... , a_n$ : increasing integers}
	\State $i \gets 1$
	\State $j \gets n$
	\While{$i < j$}	
		\State $m = \lfloor (i + j)/2 \rfloor$ 	\Comment{Middle}
		\If{$x > a_m$}
			\State $i \gets m+1$
		\Else
			\State $j \gets m$
		\EndIf

	\EndWhile
	\\
	\If {$x = a_i$}
		\State $location \gets i$		\Comment{Set location to index i}
	\Else	
		\State $location \gets i$	\Comment{Get the position where x should be inserted}
		
		\For {$i \gets n + 1 $ \bf to $ location, $ \bf decrement $ i$} 
		
		\State $arr_i \gets arr_{i-1}$ 	\Comment{Copy over values at higher index to lower index}
		\State\Comment{to occupy i+1 to n+1}
			
		\EndFor
		\State $arr_{location} \gets x$	\Comment{Insert value x into the empty }
	\EndIf
	
		
	
	\EndProcedure
	\end{algorithmic}
\end{algorithm}

\end{enumerate}
\end{document}























%------------------------------------------------------------------------------------------

\begin{comment}
    	Problem 3.14 A
    	\count255 = 1
    	\loop
    		\text Set counter to 1.
    		\ifnum\count255 < 10 
    		\advance\count255 by 1
    	\repeat
    	\end{comment}
		\begin{comment}
		\newcommand{\num}{7}	%value to search
    	\newcommand{\length}{9} %length of array  (<= DNE)
		\multido{\i=0+1}{\length}{
    	   \text Hello
    	}
		\end{comment}
    	 
    	 
    	 
    	\begin{comment}
   		\begin{enumerate}
   			\item[1.]
   			Set counter to 1.  
   			\item[2.]
   			Compare counter 1 to size of list 8				
   			\item[3.]
   			Then compare 7 to 1, the first value in the set.
   			\item[4.]
   			Since counter 1 $\leq$ 8 and values 7 $\neq$ 1 then increment counter by 1.
   			\item[5.]
   			Compare counter 2 to size of list 8				
   			\item[6.]
   			Then compare 7 to 3, the first value in the set.
   			\item[7.]
   			Since counter 2 $\leq$ 8 and values 7 $\neq$ 3 then increment counter by 1.
   		\end{enumerate}
   		\end{comment}

		\begin{comment}
  		\begin{enumerate}
   			\item[1.] %\number\count3
   			\item[2.]
   			\item[3.]
   		\end{enumerate}
		\end{comment}
%-----------------------------------------------------------------------------------
		%TODO: Work on globalization for Binary Search when I have more time
   		%Still working on implementing. For some reason, exceeds memory 
   		%\Values(\count3) should have worked;
   		\begin{comment}
		\newcounter{low}
		\setcounter{low}{0}
		\newcounter{high}
		\setcounter{high}{9}
		\count3 = 2
		%\setcounter{low}{\count3}
		%\arabic{low}
   		
   		%\Values(\count3)
   		\whiledo{\value{low} < \value{high}}{%
   			%get the index midpt
   		    \count3 = \the\value{high} + \the\value{low}
			\divide\count3 by 2
   			
	 			
   			\ifnum \num > \Values(\number\count3)  %if num > midpt -> get upper half
   		        %set low to the midpt + 1
   		    	\setcounter{low}{\count3 + 1}
   		    	higher
   		    	-- \number\count3 : \arabic{low} \arabic{high} --
   		    	
   		    \else  %if num < midpt  ->  get lower half
   		    	%set high to midpt
   		    	\setcounter{high}{\count3}
   		    	lower
   		    	-- \number\count3 : \arabic{low} \arabic{high} --
   		    	
   			\fi
   		}
   		\end{comment}